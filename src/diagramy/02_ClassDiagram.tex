\subsection{Diagram tříd I}

%% Class diagram - popis, časté použití, příklad

%%%%%%%%%%%%%%%%%%%%%%%%%%%%%%%%%%%%%%%%%%%%%%%%%%%%%%%%%%%%%%%%%%%%%%%%%%%%%%

%%Slide

\begin{frame}{Diagram tříd}

\onslide<+->Diagram tříd (Class Diagram)
\begin{itemize}[<+->]
    \item Strukturní diagram objektového modelování
    \item Většina konceptů má přímý ekvivalent v objektovém programování
    \item Různě podrobné diagramy v různých fázích vývoje 
             (třídy se rozpadají na podtřídy podle hrubosti návrhu)
\end{itemize}

\begin{columns}[T]

    \begin{column}{0.40\textwidth}
        \onslide<5->Třída  
        \begin{itemize}
            \item<6-> Jméno
            \item<6-> Artibuty
            \item<6-> Metody, Operace
            \item<6-> Viditelnost:
            \onslide<7->
            \begin{itemize}
                \item + Public
                \item \textendash ~~Private
                \item \# Protected
                \item $\sim$ ~Package
            \end{itemize}
            
        \end{itemize}
    \end{column}

    \begin{column}{0.55\textwidth}
		\onslide<5->
		\begin{figure}
			\pgfimage[width=40mm]{img/diagramy/Class/class.png}
		\end{figure}
    \end{column}

\end{columns}
\end{frame}

%%%%%%%%%%%%%%%%%%%%%%%%%%%%%%%%%%%%%%%%%%%%%%%%%%%%%%%%%%%%%%%%%%%%%%%%%%%%%%

%%Slide

\begin{frame}{Diagram tříd II}

\onslide<+->Rozsah (Scope)
\begin{itemize}
    \item<+-> Instance
    \onslide<+->
    \begin{itemize}
        \item Každá instance má vlastní kopii atributu
        \item Metoda může měnit instanci
    \end{itemize}
    
    \item<+-> Classifier
    \onslide<+->
    \begin{itemize}
        \item Odpovídá static atributům a metodám
    \end{itemize}
\end{itemize}

\onslide<+->Abstraktní třídy
\begin{itemize}[<+->]
    \item Jméno třídy je psané kurzívou
\end{itemize}

\onslide<+->Vztahy (Relationship)
\begin{itemize}
    \item<+-> Mohou být pojmenovány (přehlednost)
    \item<+-> Jména rolí (oba konce)
    \onslide<+->
    \begin{itemize}
        \item Mají viditelnost
        \item Implementace většinou jako metoda třídy
    \end{itemize}
    
    \item<+-> Arita (oba konce)
    \onslide<+->
    \begin{itemize}
        \item 0..1 Nejvýš jedna
        \item 1 Právě jedna
        \item 0..*, * Libovolně
        \item 1..* Alespoň jedna
    \end{itemize}
\end{itemize}

\end{frame}

%%%%%%%%%%%%%%%%%%%%%%%%%%%%%%%%%%%%%%%%%%%%%%%%%%%%%%%%%%%%%%%%%%%%%%%%%%%%%%

%%Slide

\begin{frame}{Diagram tříd III}

Analýza

\begin{figure}
	 \pgfimage[width=90mm]{img/diagramy/Class/analysis-level.png}
\end{figure}

\end{frame}

%%%%%%%%%%%%%%%%%%%%%%%%%%%%%%%%%%%%%%%%%%%%%%%%%%%%%%%%%%%%%%%%%%%%%%%%%%%%%%

%%Slide

\begin{frame}{Diagram tříd IV}

Implementace

\begin{figure}
	 \pgfimage[width=90mm]{img/diagramy/Class/design-level.png}
\end{figure}

\end{frame}